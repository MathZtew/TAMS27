\documentclass{article}
\usepackage{amsmath}
\usepackage[utf8]{inputenc}
\usepackage[a4paper, margin=2.8cm]{geometry}
\title{Matematisk statistik}

\begin{document}
\pagenumbering{gobble}
\maketitle
\newpage
\pagenumbering{roman}

\section{Kombinatorik}
\subsection{Definition}
\begin{itemize}
	\item{En permutation är ett arrangemang av ett antal objekt med hännsyn till deras ordning.}
	\item{En kombination är ett val av ett antal objekt utan hänsyn till deras ordning.}
\end{itemize}

\subsection{Exempel}
a) objekt A, B, C , alla permutationer:\\
	ABC, ACB, BCA, BAC, CAB, CBA
b) objekt A, B, C , alla kombinationer:\\
	ABC
	
\subsection{Exempel}
igen objekt A, B, C. Vi väljer ut 2 st.:\\
	a) alla permuatitioner AB, BA, AC, CA, BC, CB
	B) alla kombinationer AB, BC, AC\\
\\
Hur många möjligheter finns? \underline{Multiplikationsprincipen}\\
Beteckna ett experiment som utförs i k steg. Låt i för i = 1, 2, ..., k $n_i$ vara antalet sätt som steg i kan utföras på. Då är det totala antalet sätt som experimentet kan utföras på
\begin{equation*}
n = n_1, ..., n_k
\end{equation*}

Beteckning: $n! = 1 * ... * n, n \in N,0! = 1$

\end{document}